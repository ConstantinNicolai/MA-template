\pdfbookmark[1]{Abstract}{Abstract}
\begingroup
\let\clearpage\relax
\let\cleardoublepage\relax
\let\cleardoublepage\relax

% English version

\chapter*{Abstract}

% Briefly summarize the contents of your work in 150-250 words.
% \lipsum[1]
% The boom in usage of ML applications goes hand in hand with a boom in its energy costs. This necessitates a better understanding of present and future ML performance characteristics. No previous work has profiled both time and energy for inference and training while providing predictions for unknown operations. We therefore contribute a pipeline to profile all four cases and a model to predict all of them. Lastly we contribute detailed results and validations, along with obervations on general trends for optimally efficient execution. We provide a tool enabling the correct choice of target GPU and clock speed for existing and future models.



% The fast broader adoption of ML applications has caused a surge in energy requirements, necessitating a better understanding of the tradeoffs between time and energy for ML performance. Previous work was focused on time-only or inference-only studies. We contribute: (1) time and energy profiling across inference and training, (2) performance prediction trained on our profiling results and (3) graphical evalutation and validation of profiling and predictor results. Profiling results of A30 performance across several core clocks reveal an energy optimum of 900 MHz, aligning with the manufacturer base clock of 930 MHz. This work provides the tools, which enable the correct choice of target GPU and clock speed for existing and future models.


% increase in parameter count 

% tradeoffs between execution speed and energy consumption

% tradeoffs between execution speed and energy costs


% time and energy profiling across inference and training of DNN operations 


% % has caused a surge in their global energy usage


% operations-level and full DNN performance prediction


The fast broader adoption of ML applications has caused a surge in their global energy usage, necessitating a comprehensive understanding of the tradeoffs between execution speed and energy consumption. While previous work was focused on time-only or inference-only studies, we provide a more complete picture by covering a wider space of parameters. We contribute: (1) time and energy profiling across inference and training of DNN operations, (2) operations-level and full DNN performance predictions trained on our profiling results and (3) graphical evaluation and validation of profiling and predictor results. Profiling results of Nvidia A30 performance across several core clocks reveal an energy optimum of 900 MHz, aligning with the manufacturer base clock of 930 MHz. This work provides the tools which enable the correct choice of target GPU and clock speed for existing and future models.



% The fast broader adoption of ML applications has caused a surge in their energy requirements, necessitating a comprehensive understanding of the tradeoffs and costs. While previous work was focused on time-only or inference-only studies, we provide a more complete picture by covering a wider space of parameters. We contribute: (1) time and energy profiling across inference and training, (2) performance prediction trained on our profiling results and (3) graphical evaluation and validation of profiling and predictor results. Profiling results of Nvidia A30 performance across several core clocks reveal an energy optimum of 900 MHz, aligning with the manufacturer base clock of 930 MHz. This work provides the tools which enable the correct choice of target GPU and clock speed for existing and future models.



\newpage

% German version

\begin{otherlanguage}{ngerman}
\pdfbookmark[1]{Zusammenfassung}{Zusammenfassung}
\chapter*{Zusammenfassung}

% Die rasche Verbreitung von ML Anwendungen hat zu einem starken Anstieg der Energiekosten geführt und damit ein besseres Verständnis der 
% Tradeoffs zwischen Zeit und Energie notwendig gemacht. Der Fokus vorheriger Arbeiten liegt auf reinen Zeit oder reinen Inference Studien. Unsere Contributions sind: (1) Zeit- und Energieprofiling für Inference und Training, (2) Vorhersagen trainiert auf den Profiling Ergebnissen und (3) eine grafische Auswertung und Validierung der Profiling und Verhersage Ergebnisse. Messresultate der A30 Performance über einige GPU Clocks zeigen ein Energieoptimum bei 900 MHz, welches sich mit dem Standard Base Clock von 930 MHz deckt. Diese Arbeit gibt uns die Werkzeuge um die richtige GPU und den richtigen Clock Speed für aktuelle und zuküftige Modelle zu wählen.


Die rasche Verbreitung von ML Anwendungen hat zu einem starken Anstieg ihrer globalen Energiekosten geführt und damit ein umfassendes Verständnis der Tradeoffs zwischen Laufzeit und Energieverbrauch notwendig gemacht. Der Fokus vorheriger Arbeiten liegt auf reinen Zeit- oder reinen Inference-Studien. Hierauf bauen wir auf, indem wir eine breitere Spanne an Parametern abdecken um ein vollständigeres Bild der Lage abzugeben. Unsere Contributions sind: (1) Zeit- und Energieprofiling für Inference und Training von DNN Operations, (2) Vorhersagen für individuelle Operations und volle DNNs trainiert auf den Profiling Ergebnissen und (3) eine grafische Auswertung und Validierung der Profiling und Vorhersage Ergebnisse. Messresultate der Nvidia A30 Performance über einige GPU Clocks zeigen ein Energieoptimum bei 900 MHz, welches sich mit dem Standard Base Clock von 930 MHz deckt. Diese Arbeit stellt die Werkzeuge zur Verfügung um die richtige GPU und den richtigen Clock Speed für aktuelle und zukünftige Modelle zu wählen.
\end{otherlanguage}

\endgroup

\vfill
