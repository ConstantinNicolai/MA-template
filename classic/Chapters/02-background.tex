\chapter{Background}\label{chap:background}




\section{Random Forest Regressor}

In order to introduce the workings of a random forest regressor model, we first need define to a number of foundational concepts: regression, decision trees, bootstrapping and bagging. On this basis, we can describe the concept of a random forest. \\
Regression is a supervised learning technique approximating a function to a continuous target variable from a set of input output pairs. In the case of random forests, this function is learned by minimizing the mean squared error over an ensemble of decision trees \citep{hastie_elements_2009}[p. 10]. \\
A decision tree is a predictor which divides up the input space in order to provide specific predictions for each region. In the tree, at each decision node one feature and a corresponding threshold are chosen to split the input space into two regions. This feature and threshold are chosen by considering all features and many thresholds and identifying the lowest MSE combination. At each leaf node, a prediction value is assigned based on the data points within that region \citep{hastie_elements_2009}[p. 307]. \\
Bootstrapping describes the process of sampling with replacement from the training dataset. For each tree in the ensemble, a new training subset is created by randomly sampling data points from the full training dataset, explicitly allowing for duplicates. This results in each tree being trained on a slightly different subset of the training dataset, while preserving the overall data distribution across the ensemble of trees. \\
The ensemble technique of aggregating the results from the independently trained trees is called bagging. For regression tasks, the individual predictions are averaged, which serves to reduce model variance and improve robustness across the ensemble. \\
The only step necessary to move from standard bagging to a random forest model, is to adapt the process of building the decision trees. Instead of considering all features at every decision node, in a random forest only a smaller random subset of features is considered at each node. Introducing randomness within the building of the trees increases variety between the trees. This reduced correlation between individual trees is the core improvement over the standard bagging approach. The individual trees being more independent results in reduced variance in the average across the ensemble \citep{hastie_elements_2009}[p. 588]. \\
The twofold introduction of randomness through bootstrapping and the random forest approach is the key to the overall strengths of the random forest model in strong robustness, little overfitting and good generalization.




\section{XGBoost Regressor}

XGBoost is another tree based ensemble method for regression. In contrast to the random forest regressor, it builds its trees sequentially instead of concurrently like the random forest does. \\
Classic gradient boosting proceeds by building an initial tree to fit the original targets. When using MSE, this is typically the mean of the targets. From then on, it works iteratively. Each subsequent tree is trained to predict the negative partial derivative of the loss with respect to the current prediction. The trees are then added to the ensemble and weighted with a learning rate which determines the contribution of each step and controls the convergence behavior of the model \cite{chen_xgboost_2016}. \\
The primary drawback of classic gradient boosting is its tendency to overfit, especially for larger ensembles of trees. XGBoost builds upon classic gradient boosting in order to preserve its strengths and improve its generalization. \\
One improvement is its inclusion of regularization. It uses a leaf weight penalty term which encourages sparsity by driving leaf weights to zero. It also introduces a minimum loss reduction threshold that controls node splitting. In practice that means, if a split does not improve the loss function by at least the threshold amount, no split occurs \cite{chen_xgboost_2016}. XGBoost also often uses a small learning rate which requires a large number of trees, but improves the training stability. Another extension is early stopping. It evaluates performance on a held-out validation set during training, stopping if no further improvement occurs. The combination of more stable training and early stopping reduces overfitting.




\section{Coefficient of Determination}


The coefficient of determination, denoted as $R^2$, is a statistical measure used to evaluate the performance of regression models. It measures the proportion of the variance in the dependent variable that is predictable from the independent variables. 
\begin{equation}
R^2 = 1 - \frac{\sum_{i=1}^n (y_i - \hat{y}_i)^2}{\sum_{i=1}^n (y_i - \bar{y})^2}
\end{equation}
where $y_i$ are the observed values, $\hat{y}_i$ are the predicted values and $\bar{y}$ is the mean of the observed values. \\
An $R^2$ score of $1$ indicates perfect predictions. A score of $0$ indicates the predictor performs equally to predicting the mean of the training target values. If the model performs worse than predicting the mean, the score ranges below $0$. \\
Before the training process of any model begins, the dataset is split into a training set and a test set. The test set is held out for the final evaluation of the predictive performance on unseen data.


\subsection{Cross-Validation $\overline{R^2}$}
In $k$-fold cross-validation, the training set is partitioned into k subsets called folds. The regressor is trained on $k-1$ folds and the $R^2$ score is evaluated on the remaining fold. This process is repeated $k$ times across all folds. The $k$ scores are then averaged into a CV $\overline{R^2}$. This process only simulates a holdout set without actually interacting with the test set. Therefore it is a measure of how well the model works compared to other models and while it can give an indication of the predictive performance, it is not a conclusive measure its performance.

\subsection{Test Set $R^2$}
In contrast, the test set $R^2$ score is a measure of the models real-world performance, since its evaluation is performed on a true holdout set. It provides a conclusive measure of the models generalization and its real-world performance.
