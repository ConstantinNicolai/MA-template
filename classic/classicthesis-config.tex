% *****************************************************************************
% 0. Set the encoding of your files. UTF-8 is the only sensible encoding
% nowadays. If you can't read äöüßáéçèê∂åëæƒÏ€ then change the encoding setting
% in your editor, not the line below. If your editor does not support utf8 use
% another editor!
% *****************************************************************************
\PassOptionsToPackage{utf8}{inputenc}
\usepackage{inputenc}

\PassOptionsToPackage{T1}{fontenc} % T2A for cyrillics
\usepackage{fontenc}

% *****************************************************************************
% Personal data and user ad-hoc commands (insert your own data here)
% *****************************************************************************
\newcommand{\myTitle}{Your Title Here\xspace}
\newcommand{\myName}{Your Name Here\xspace}
\newcommand{\myMatNr}{00000 \xspace}
\newcommand{\myPlaceOfBirth}{City, Country\xspace}
\newcommand{\myProf}{Holger Fröning\xspace}
\newcommand{\mySupervisor}{Jon Doe\xspace}
\newcommand{\myLocation}{Heidelberg\xspace}
\newcommand{\mySubmissionDate}{Day/Month/Year Here\xspace}
\newcommand{\myVersion}{\classicthesis}



% *****************************************************************************
% 1. Configure classicthesis for your needs here, e.g., remove "drafting" below
% in order to deactivate the time-stamp on the pages
% (see ClassicThesis.pdf for more information):
% *****************************************************************************
\PassOptionsToPackage{
	drafting=false,    % print version information on the bottom of the pages
	tocaligned=false, % the left column of the toc will be aligned (no indentation)
	dottedtoc=false,  % page numbers in ToC flushed right
	eulerchapternumbers=true, % use AMS Euler for chapter font (otherwise Palatino)
	linedheaders=false,       % chaper headers will have line above and beneath
	floatperchapter=true,     % numbering per chapter for all floats (i.e., Figure 1.1)
	eulermath=false,  % use awesome Euler fonts for mathematical formulae (only with pdfLaTeX)
	beramono=true,    % toggle a nice monospaced font (w/ bold)
	palatino=true,    % Use Palatino font
	style=arsclassica % classicthesis, arsclassica
}{classicthesis}


% *****************************************************************************
% 2. Setup, finetuning, and useful commands
% *****************************************************************************
\providecommand{\mLyX}{L\kern-.1667em\lower.25em\hbox{Y}\kern-.125emX\@}
\newcommand{\ie}{i.\,e.}
\newcommand{\Ie}{I.\,e.}
\newcommand{\eg}{e.\,g.}
\newcommand{\Eg}{E.\,g.}
\newcommand{\MatVal}[1]{\bm{\mathit{#1}}} % use to typeset matrix variables
% *****************************************************************************


% *****************************************************************************
% 3. Loading some handy packages
% *****************************************************************************
\PassOptionsToPackage{ngerman,american}{babel} % change this to your language(s), main language last
% Spanish languages need extra options in order to work with this template
%\PassOptionsToPackage{spanish,es-lcroman}{babel}
\usepackage{babel}

\usepackage{csquotes}
\PassOptionsToPackage{%
	backend=biber,bibencoding=utf8, %instead of bibtex
	% backend=bibtex8,bibencoding=ascii,%
	language=auto,%
	style=numeric-comp,%
	%style=authoryear-comp, % Author 1999, 2010
	%bibstyle=authoryear,dashed=false, % dashed: substitute rep. author with ---
	sorting=nyt, % name, year, title
	maxbibnames=10, % default: 3, et al.
	%backref=true,%
	natbib=true % natbib compatibility mode (\citep and \citet still work)
}{biblatex}
\usepackage{biblatex}

\PassOptionsToPackage{fleqn}{amsmath}       % math environments and more by the AMS
\usepackage{amsmath}

% *****************************************************************************
% General useful packages
% *****************************************************************************
\usepackage{verbatimbox}
\usepackage{bm}
\usepackage{mathdots}
\usepackage{mathtools}
\usepackage{tikz}
\usetikzlibrary{trees, matrix, positioning, patterns, shapes, shadows, shapes.arrows, arrows.meta, shapes.multipart, decorations.pathreplacing, calc, tikzmark, shapes.geometric}
\usepackage[binary-units]{siunitx}
\usepackage{lipsum}
\usepackage{graphicx} %
\usepackage{scrhack} % fix warnings when using KOMA with listings package
\usepackage{xspace} % to get the spacing after macros right
\PassOptionsToPackage{printonlyused,smaller}{acronym}
\usepackage{acronym} % nice macros for handling all acronyms in the thesis
\def\bflabel#1{{\acsfont{#1}\hfill}}
\def\aclabelfont#1{\acsfont{#1}}
% *****************************************************************************
\usepackage{pgfplots} % External TikZ/PGF support (thanks to Andreas Nautsch)
\pgfplotsset{compat=1.18}
\usepgfplotslibrary{units}
%\usetikzlibrary{external}
%\tikzexternalize[mode=list and make, prefix=ext-tikz/]
% *****************************************************************************


% *****************************************************************************
% 4. Setup floats: tables, (sub)figures, and captions
% *****************************************************************************
\usepackage{tabularx} % better tables
\setlength{\extrarowheight}{3pt} % increase table row height
\newcommand{\tableheadline}[1]{\multicolumn{1}{l}{\spacedlowsmallcaps{#1}}}
\newcommand{\myfloatalign}{\centering} % to be used with each float for alignment
\usepackage{subfig}
% *****************************************************************************


% *****************************************************************************
% 5. Setup code listings
% *****************************************************************************
\usepackage{listings}
%\lstset{emph={trueIndex,root},emphstyle=\color{BlueViolet}}%\underbar} % for special keywords
\lstset{language=C++,
	morekeywords={float2,float3,float4,__device__,__forceinline__,__host__,__global__},
	keywordstyle=\color{RoyalPurple},%\bfseries,
	basicstyle=\footnotesize\ttfamily,
	identifierstyle=\color{WildStrawberry},
	commentstyle=\color{SeaGreen}\ttfamily,
	stringstyle=\ttfamily,
	numbers=left,%left,%
	numberstyle=\scriptsize,%\tiny
	stepnumber=1,
	numbersep=8pt,
	showstringspaces=false,
	breaklines=true,
	%frameround=ftff,
	frame=single,
	belowcaptionskip=.75\baselineskip,
  captionpos=b
	%frame=L
}
% *****************************************************************************

\usepackage{classicthesis}

% *****************************************************************************
% Fine-tune hyperreferences (hyperref should be called last)
% *****************************************************************************
\hypersetup{%
	%draft, % hyperref's draft mode, for printing see below
	colorlinks=true, linktocpage=true, pdfstartpage=3, pdfstartview=FitV,%
	% uncomment the following line if you want to have black links (e.g., for printing)
	%colorlinks=false, linktocpage=false, pdfstartpage=3, pdfstartview=FitV, pdfborder={0 0 0},%
	breaklinks=true, pageanchor=true,%
	pdfpagemode=UseNone, %
	% pdfpagemode=UseOutlines,%
	plainpages=false, bookmarksnumbered, bookmarksopen=true, bookmarksopenlevel=1,%
	hypertexnames=true, pdfhighlight=/O,%nesting=true,%frenchlinks,%
	urlcolor=CTurl, linkcolor=CTlink, citecolor=CTcitation, %pagecolor=RoyalBlue,%
	%urlcolor=Black, linkcolor=Black, citecolor=Black, %pagecolor=Black,%
	pdftitle={\myTitle},%
	pdfauthor={\textcopyright\ \myName, Heidelberg University, Institute of Computer Engineering},%
	pdfsubject={},%
	pdfkeywords={},%
	pdfcreator={pdfLaTeX},%
	pdfproducer={LaTeX with hyperref and classicthesis}%
}


% *****************************************************************************
% Setup autoreferences (hyperref and babel)
% *****************************************************************************
% There are some issues regarding autorefnames
% http://www.tex.ac.uk/cgi-bin/texfaq2html?label=latexwords
% you have to redefine the macros for the
% language you use, e.g., american, ngerman
% (as chosen when loading babel/AtBeginDocument)
% *****************************************************************************
\makeatletter
\@ifpackageloaded{babel}%
{%
	\addto\extrasamerican{%
		\renewcommand*{\figureautorefname}{Figure}%
		\renewcommand*{\tableautorefname}{Table}%
		\renewcommand*{\partautorefname}{Part}%
		\renewcommand*{\chapterautorefname}{Chapter}%
		\renewcommand*{\sectionautorefname}{Section}%
		\renewcommand*{\subsectionautorefname}{Section}%
		\renewcommand*{\subsubsectionautorefname}{Section}%
	}%
	\addto\extrasngerman{%
		\renewcommand*{\paragraphautorefname}{Absatz}%
		\renewcommand*{\subparagraphautorefname}{Unterabsatz}%
		\renewcommand*{\footnoteautorefname}{Fu\"snote}%
		\renewcommand*{\FancyVerbLineautorefname}{Zeile}%
		\renewcommand*{\theoremautorefname}{Theorem}%
		\renewcommand*{\appendixautorefname}{Anhang}%
		\renewcommand*{\equationautorefname}{Gleichung}%
		\renewcommand*{\itemautorefname}{Punkt}%
	}%
	% Fix to getting autorefs for subfigures right (thanks to Belinda Vogt for changing the definition)
	\providecommand{\subfigureautorefname}{\figureautorefname}%
}{\relax}
\makeatother

\listfiles
